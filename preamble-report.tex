% Option for LaTex documents
% Linus Conrad
%-------------------------

% BASIC SETUP
% these packages need to be loaded before others
%-------------------------

% language selection for hyphenation etc, use \selectlanguage{x} in a document to change
% ngerman is German with the reformed orthographic rules
\usepackage[UKenglish]{babel}

% KP font family, full family with math support with old fashioned feel
% requires no further setup
% may need to be loaded with option 'notextcomp' if combined with other fonts
%\usepackage[lighttext]{kpfonts}

% The font, chosen because it is easy to match in ggplot and has full math support
\usepackage{tgpagella}% Palladio needs more leading (space between lines)


% Typewriter font for listings and Sequences
%\usepackage[varqu,varl]{inconsolata} % sans serif style fixed width font
\usepackage[zerostyle=d]{newtxtt} % serif style style fixed width font

% required for full glyph support
\usepackage[LGR,T1]{fontenc}

% utf-8 encoding for support of umlaut etc without escape characters in .tex files
\usepackage[utf8]{inputenc}
% required package to set typographically correct quotation marks, needed by babel and biber
\usepackage{csquotes}
% needed not to mess up greek letters in titles etc
\usepackage{textalpha}

% this gets rid of annoying "substituted fontsize" errors and enables seamless scaling
\usepackage{anyfontsize}

% TYPESETTING & STYLE
%-------------------------
% sets the space between lines
\usepackage{setspace}
    \setstretch{1.3}	
% necessary to specify coloured text
\usepackage{xcolor}
% Inkscape may throw error for not having this 
\usepackage{transparent}
%prevents widows and orphans
\usepackage[all]{nowidow}

% further options to the KOMA class
% These determine the page layout and define font styles
% for defined parts of the doc structure (header, maint ext etc)
% ------------------------
\KOMAoptions{DIV = 10,
   headings = standardclasses, %sets disposition to roman (see below) and adjusts sizes of  disposition objects like in the standard class
   % chapterprefix = true, % only for books
   headsepline = true,  % sets a line in the header
   %footsepline = true,
   abstract = true, % adds a header to the abstract
   %BCOR = 2cm  % binding correction
   draft = true  % draft mode
}
	
% Koma options for typesetting the ToC
%\KOMAoptions{listof = totoc}

% Komaoption of the ToC
\KOMAoption{toc}{
  bib%, %  the bibliography has an unnumbered entry in the table of contents
  %listof, %  makes a list of figures into toc
}

% Komaoptions for sectioning headlines
%\KOMAoption{headings}{
%    big,        % headlines are less mighty (chapter headlines are enlarged below below)
%    optiontohead  % optional short title gets written into header, full title goes to ToC
%    }

% Give instructions how to set the font particular elements (captions and more)
%\setkomafont{sectioning}{\rmfamily\bfseries\boldmath} % example


% Koma options for typesetting captions
\KOMAoption{captions}{
  centeredbeside,  % captions set to the side of a float appear vertically centered
  outerbeside,     % captions appear on the inner side in a double sided document
	}

%\renewcommand*{\captionformat}{ }  % this determines the figure label separator
\setkomafont{captionlabel}{\normalsize\bfseries}
\addtokomafont{caption}{\footnotesize\rmfamily}
\setcapindent{0em} % sets and indent on captions (removes "hanging" caption) 

\setkomafont{descriptionlabel}{\rmfamily\bfseries}

\setkomafont{subsubsection}{\centering\itshape\mdseries\large}

%KOMA dictum environment
\renewcommand{\dictumwidth}{0.5\textwidth}
\setkomafont{dictum}{\normalfont\normalcolor\rmfamily\small}

%\setkomafont{pageheadfoot}{\scfamily} % footer
%\setkomafont{pageheadfoot}{\rmfamily\scshape}  

%\setkomafont{pagenumber}{\sffamily} % pagenumber
%\setkomafont{pagenumber}{\rmfamily} 
                    
% see documentation, this removes justified headings as defined by KOMA                            
\let\raggedsection\relax

% Packages adding functionality
% -----------------------------
% comprehensive support for mathematical glyphs
\usepackage{amsmath}

% automatical typesetting of units
\usepackage[separate-uncertainty=true,
    multi-part-units=single]{siunitx}
	\sisetup{detect-all}
	\DeclareSIUnit\molar{M}
	\DeclareSIUnit{\calorie}{cal}

% typographical tables without vertical lines
\usepackage{booktabs}

% provides table environment adjusted to the width of the text, supports linebreaks in cells
\usepackage{tabularx}

% improoves typessing, hanging punctuation and adjustment of glyph width etc
\usepackage{microtype}

% provides compact list environments that waste less space
\usepackage{paralist}

% Required for inserting any graphics
\usepackage{graphicx}
	\DeclareGraphicsExtensions{.pdf_tex,.pdf,.png,.jpg}
	%declare additional paths for all chapters
	\graphicspath{	
	{./figure/},
	{./figure/export/}}

% set how much text to fig ratio is accepted befre creating a figure only page
\renewcommand{\floatpagefraction}{.85}
\renewcommand{\topfraction}{.8}
\renewcommand{\bottomfraction}{.8}

% allow latex to make shorter pages if necesary to prevent awkward sections
% starting at the very bottom of a page
% disable this in double sided docs
\raggedbottom

% support for non unix-style path to graphics
%\usepackage{grffile}

% allows an alternative \clearpage command that does not cause a pagebreak but forces to place all floats
\usepackage{afterpage}

% reference sections or figures automatically with \cref{#label}
\usepackage[english]{cleveref}
% this gets rid of the braket around the number
\creflabelformat{equation}{#2#1#3}

% making glossaries, has to be set after hyperref!
\usepackage[nonumberlist]{glossaries}
 
% REFERENCING
%-------------------------
\usepackage[backend=biber,
  style = authoryear, %numeric or authoryear
  backref = false,
%  refsegment = chapter,
  sorting = nyt,
  %natbib=true,
  date = year,  % drops any other info on date of publication from references
  maxcitenames = 2,
  %ibidtracker = true,
  maxbibnames = 4,  
  giveninits = true,
  %terseinits = true,
  uniquelist = false,
  uniquename = false 
  ]{biblatex}


% remove unwanted entries
\DeclareSourcemap{
\maps[datatype=bibtex, overwrite]{
	\map{\step[fieldset=language, null]
		\step[fieldset=notes, null]
		\step[fieldset=urldate, null]
		\step[fieldset=url, null]
		\step[fieldset=doi, null]
		\step[fieldset=issn, null]
		\step[fieldset=month, null]
		\step[fieldset=editor, null]
	}
}
}

% This line is important!
%\addbibresource{samplebibliography.bib}

% set the font for the bibliography to SF
\renewcommand*{\bibfont}{\footnotesize}
	      
% removes a superflous comma after first author in bibliography      	 
%\renewcommand*{\revsdnamepunct}{} 
%\newcommand{\tempmaxup}[1]{\def\blx@maxcitenames{\blx@maxbibnames}#1} 

% Change the spacing in knitR listings to single
\renewenvironment{Shaded}{\singlespacing\begin{snugshade}}{\end{snugshade}}


% layout package to print the size of some values
% I need to know this to export the figures at appropiate size to scale the fonts correctly
%\usepackage{layouts}

%this command suppresses hbox warnings for small violations
\hfuzz=4pt 

% this determines how numbered sections appear
\setcounter{secnumdepth}{2}
\setcounter{tocdepth}{1}
\setcounter{section}{0}


